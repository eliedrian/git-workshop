\documentclass[17pt]{beamer}

\mode<presentation>

\usetheme{moloch}

\usepackage{parskip}
\usepackage{hyperref}

\title{Git}
\subtitle{in practice}

\begin{document}

\maketitle

\begin{frame}
  \centering
  \includegraphics[height=\textheight]{git-meme}
\end{frame}

\begin{frame}{Install \texttt{git}}
  \url{https://gitforwindows.org/}

  \url{https://git-scm.com/downloads}
\end{frame}

\begin{frame}
  \centering
  \includegraphics[height=\textheight]{git-for-windows}
\end{frame}

\begin{frame}{First-time setup}
  \begin{itemize}
    \item hu u?
    \item text editor
  \end{itemize}
\end{frame}

\begin{frame}{hu u?}
  \texttt{git config --global user.name <your-name>}

  \texttt{git config --global user.email <your-email>}
\end{frame}

\begin{frame}
  \centering
  \includegraphics[height=\textheight]{vscode-meme}
\end{frame}

\begin{frame}{VSCode}
  \texttt{git config --global core.editor "code --wait"}
\end{frame}

\begin{frame}{Verify}
  \texttt{git config --list}
\end{frame}

\begin{frame}{In or out?}
  \begin{itemize}
    \item tracked
    \item untracked
  \end{itemize}
\end{frame}

\begin{frame}{Tracked file states}
  \begin{itemize}
    \item modified
    \item staged
    \item committed
  \end{itemize}
\end{frame}

\begin{frame}{Create a repo}
  In project folder:

  \texttt{git init}
\end{frame}

\begin{frame}{Cloning a repo}
  \texttt{git clone <repo-url>}
\end{frame}

\begin{frame}{Status}
  \texttt{git status}
\end{frame}

\begin{frame}{Track}
  Stage modifications:

  \texttt{git add <filename>}
\end{frame}

\begin{frame}{Commit}
  \texttt{git commit}

  \texttt{git commit -m <message>}
\end{frame}

\begin{frame}
  \centering
  \includegraphics[height=\textheight]{commit-meme}
\end{frame}

\begin{frame}{Diff}
  \texttt{git diff}

  \texttt{git diff --staged}
\end{frame}

\begin{frame}{Delete}
  \texttt{git rm}
\end{frame}

\end{document}
